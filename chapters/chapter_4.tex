\chapter{Analysewerkzeuge für Netcode und Debugging}

Zur Untersuchung netzwerkbasierter Spielmechaniken wie Client-Side Prediction, Lag Compensation oder Interpolation ist der Einsatz geeigneter Analysewerkzeuge essenziell. Ziel ist es, sowohl das Verhalten dieser Mechaniken unter verschiedenen Bedingungen zu verstehen, als auch deren Einfluss auf das Spielgefühl sichtbar und messbar zu machen.

Im Rahmen dieser Arbeit soll ein System verwendet werden, bestehend aus einem Debug-Overlay und einer integrierten Ingame-Konsole. Dieses Werkzeug ermöglicht es, netzwerkrelevante Parameter -- wie beispielsweise Interpolationszeiten, künstliche Latenz oder Glättungsalgorithmen -- zur Laufzeit zu verändern. Alternativ wird dies nach herkömmlichem Ablauf über Netzwerkvariablen im Unity-Editor  \\
So lassen sich spezifische Szenarien gezielt nachstellen und deren Auswirkungen visuell nachvollziehen.

Neben dieser Eigenentwicklung existieren auch externe Lösungen, wie etwa der Unity Profiler mit Netcode-Unterstützung oder Werkzeuge von Drittanbietern wie Photon Fusion Analyzer. Diese bieten detaillierte technische Einblicke, sind jedoch oft nicht für eine interaktive Analyse innerhalb des Spiels ausgelegt.

Für die Konsole, die auch in Spielen wie Counter Strike oder anderen Spielen existiert um ohne Benutzeroberfläche Einstellungen zu ändern, wird hier aus dem Unity Asset Store bezogen. \cite{}
Mit diesem Asset lassen sich Konsolen-Kommandos im Quellcode registrieren, die den Spielfluss manipulieren können. Außerdem ist auch das Auslesen der Debugging-Konsole möglich wenn man im Build ist, welcher keine integrierte Konsole besitzt wie der normale Editor.

Das entwickelte Debug-System hingegen erlaubt eine tiefere Integration in den Entwicklungsprozess: Es zeigt Zustände wie die aktuelle Netzwerkverzögerung, Tick-Synchronisation oder Prediction-Fehler direkt im Spiel an. Ergänzend dazu erlaubt die Konsole das Aktivieren und Deaktivieren einzelner Netcode-Komponenten oder das dynamische Nachjustieren von Parametern -- ohne das Spiel neu starten zu müssen.

Durch diese Flexibilität ist das System besonders geeignet für die iterative Entwicklung und Bewertung von Netcode-Strategien und stellt daher das zentrale Analysewerkzeug dieser Arbeit dar.


