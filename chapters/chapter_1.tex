\chapter{Einleitung}
\label{chapter_1}
\newcommand{\keyword}[1]{\textbf{#1}}
\newcommand{\tabhead}[1]{\textbf{#1}}
\newcommand{\code}[1]{\texttt{#1}}
\newcommand{\file}[1]{\texttt{\bfseries#1}}
\newcommand{\option}[1]{\texttt{\itshape#1}}

Moderne Echtzeitanwendungen, insbesondere im Bereich der Computerspiele, stellen hohe Anforderungen an die zugrunde liegende Netzwerkarchitektur. Die Synchronisation von Spielzuständen, die Minimierung von Latenzzeiten sowie ein robuster Umgang mit Paketverlust und Jitter sind entscheidend für eine positive Nutzererfahrung. Multiplayer-Systeme, wie sie etwa mit Unity entwickelt werden, müssen deshalb nicht nur funktional, sondern auch leistungsfähig und fehlertolerant sein.

Die vorliegende Arbeit beschäftigt sich mit den theoretischen und technischen Grundlagen der Netzwerkkommunikation in Mehrspielerumgebungen. Anhand eines prototypischen Projekts in Unity werden exemplarisch zentrale Mechanismen wie Reconciliation, Interpolation sowie die Kommunikation im Client-Server-Modell umgesetzt und analysiert.

Im Folgenden werden zunächst die Motivation und Zielsetzung der Arbeit erläutert. Anschließend wird der strukturelle Aufbau der Arbeit vorgestellt.

\section{Motivation}
Multiplayer-Komponenten sind aus modernen digitalen Spielen kaum mehr wegzudenken. Sowohl kompetitive als auch kooperative Mehrspieler-Modi sind in einer Vielzahl von Genres – von taktischen First-Person-Shootern über Survival-Games bis hin zu Rennspielen – zum Standard geworden. Mit dieser Entwicklung steigen auch die Anforderungen an die zugrunde liegenden Netcode-Systeme, insbesondere hinsichtlich Präzision, Konsistenz und Latenzkompensation.

Während große Entwicklerstudios in der Lage sind, umfangreiche Netzwerkinfrastrukturen zu realisieren, stellt der zuverlässige und performante Betrieb eines Netcodes für kleinere oder unabhängige Entwicklerstudios eine erhebliche Herausforderung dar. Die Komplexität entsprechender Systeme führt häufig dazu, dass Mehrspielerfunktionen nur rudimentär implementiert oder zugunsten anderer Spielmechaniken zurückgestellt werden.

Auch bei großbudgetierten Produktionen bleibt die Qualität der Netzwerkmechanismen regelmäßig Gegenstand öffentlicher Kritik. Spielende bemängeln unter anderem sogenannte \emph{Desynchronisationen} (Desyncs), bei denen Spielerinteraktionen nicht konsistent übermittelt werden – etwa in Form von „Kill Trades“ oder dem sprichwörtlichen „Sterben um die Ecke“. Solche Phänomene sind Ausdruck komplexer zeitlicher und logischer Inkonsistenzen in der Server-Client-Kommunikation.

Ein aktuelles Beispiel für einen alternativen technologischen Ansatz ist die Einführung eines sogenannten Subtick-Systems im Titel \emph{Counter-Strike 2}. Anstelle fester Tickintervalle sollen Eingaben auf Basis präziser Zeitstempel verarbeitet werden. Dieses Konzept verspricht eine höhere Eingabegenauigkeit und ein konsistenteres Spielerlebnis, wirft aber gleichzeitig Fragen zur tatsächlichen Wirksamkeit und Praxistauglichkeit solcher Systeme auf.

Die hohe technische Komplexität sowie die Vielzahl konkurrierender Anforderungen machen deutlich, dass es sich bei Netcode-Systemen nicht um triviale Umsetzungen handelt. Vielmehr stellt sich die Frage, inwiefern es überhaupt möglich ist, eine universell optimale Lösung zu entwickeln – oder ob es sich um ein inhärent trade-off-basiertes System handelt, in dem stets zwischen Genauigkeit, Latenz, Stabilität und Ressourcenverbrauch abgewogen werden muss.

\section{Zielsetzung}
In dieser Arbeit sollen Netcode und darauf aufbauende Netcode Mechanismen auf theoretischer und praktischer Grundlage beleuchtet werden und in einem prototypischen Spiel integriert werden.
Da die reine Umsetzung keinen großen akademischen Mehrwert bietet, sollen Strategien entworfen werden, die zur Visualisierung und damit der Verbesserung des Verständnisses der Mechaniken dienen.
Es sollen des weiteren verschiedene Szenarien untersucht werden, die auch von Idealbedingungen abweichen können.

\section{Redraft}

Multiplayer-Funktionalität ist in modernen Videospielen nahezu allgegenwärtig – sei es in Form kompetitiver Mehrspieler-Modi, kooperativer Spielvarianten oder persistenter Online-Welten. Titel aus unterschiedlichsten Genres – darunter strategische First-Person-Shooter, Survival-Spiele oder Rennsimulationen – stellen hohe Anforderungen an die Netzwerksynchronisation zwischen den Spielteilnehmern. Im Zentrum dieser Anforderungen steht ein performantes und präzises Netcode-System, das eine konsistente Spielerfahrung gewährleistet.

Die technische Umsetzung eines solchen Systems bringt erhebliche Herausforderungen mit sich. Besonders kleinere Entwicklerstudios und Indie-Teams sehen sich oftmals gezwungen, auf Mehrspieler-Komponenten gänzlich zu verzichten oder diese stark zu reduzieren, da die Implementierung eines zuverlässigen Netcodes umfassende Kenntnisse in Bereichen wie Netzwerktechnologien, Latenzkompensation und Serverarchitektur erfordert.

Auch bei großen Produktionen (AAA-Titeln) ist die Kritik an unzureichender Netcode-Qualität regelmäßig Gegenstand von Diskussionen in der Fachpresse und Community. Häufig wird dabei der Vorwurf laut, dass zur Kostenreduktion auf leistungsschwächere Infrastruktur – beispielsweise Server mit niedriger Tickrate – zurückgegriffen wird. Die daraus resultierenden Probleme wie asynchrone Spielerinteraktionen, sogenannte \emph{Desynchronisationen} (Desyncs), oder Phänomene wie das „Sterben um Ecken“ und \emph{Kill Trades} wirken sich spürbar negativ auf das Spielerlebnis aus.

Ein prominentes Beispiel für die aktuelle Weiterentwicklung von Netcode-Systemen bietet \emph{Counter-Strike 2}, der Nachfolger von \emph{CS:GO}. Anstelle einer Modernisierung des klassischen Tick-basierten Modells wurde ein Subtick-System eingeführt, das Eingaben nicht mehr an feste Tickintervalle bindet, sondern mit präzisem Zeitstempel verarbeitet. Obwohl die Grundidee eines solchen Systems nicht neu ist, stellt die konsequente Umsetzung in einem Mainstream-Titel einen signifikanten technologischen Schritt dar. Ziel dieser Architektur ist eine verbesserte Präzision der Eingabeverarbeitung sowie eine höhere Synchronität zwischen Server und Client.

Angesichts dieser Entwicklungen stellt sich die Frage, ob die oftmals kritisierte Qualität des Netcodes primär auf Entwicklungsentscheidungen zurückzuführen ist – oder ob es sich vielmehr um ein inhärent komplexes Problemfeld handelt, das durch eine Vielzahl technischer, ökonomischer und gestalterischer Faktoren beeinflusst wird und kaum in seiner Gesamtheit zu „perfektionieren“ ist.

