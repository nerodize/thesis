\documentclass[12pt,a4paper]{article}
\usepackage[utf8]{inputenc}
\usepackage[ngerman]{babel}
\usepackage[T1]{fontenc}
\usepackage[colorlinks=true, linkcolor=black, urlcolor=blue, citecolor=black]{hyperref}
\usepackage{lmodern}
\usepackage{setspace}
\usepackage{titlesec}

\title{Gliederung der Abschlussarbeit:\\
Netcode-Mechaniken in Echtzeitanwendungen}
\author{Alexander Seitz}
\date{\today}

\begin{document}

\maketitle
\tableofcontents
\newpage

\section{Einleitung}
\subsection{Motivation}
\subsection{Zielsetzung der Arbeit}
\subsection{Aufbau der Arbeit}

\clearpage

\section{Theoretische Grundlagen}
\subsection{Netzwerke in Echtzeitanwendungen}
\subsubsection{Latenz, Paketverlust, Jitter, Tickrate}
\subsection{Architekturen in Multiplayer-Systemen}
\subsubsection{Client-Server-Modell}
\subsubsection{Peer-to-Peer-Architektur}
\subsubsection{Authority-Konzepte: Client vs. Server}
\subsection{Begriffserklärungen}
\subsubsection{Tickrate und Input Delay}
\subsubsection{Reconciliation, Prediction, Interpolation}

\clearpage

\section{Netcode-Mechaniken im Detail}
\subsection{Client-Side Prediction}
\subsubsection{Funktionsweise}
\subsubsection{Vorteile und Risiken (Rubberbanding, Desyncs)}
\subsection{Interpolation}
\subsubsection{Zielsetzung: Flüssige Darstellung}
\subsubsection{Techniken: LERP, Snapshot Buffering}
\subsection{Lag Compensation}
\subsubsection{Rewind-Mechaniken bei Treffererkennung}
\subsubsection{Trade-offs: Fairness vs. Komplexität}

\clearpage

\section{Umsetzung des Prototyps in Unity}
\subsection{Projektstruktur und Designentscheidungen}
\subsubsection{Aufbau der Zielumgebung}
\subsubsection{Verzicht auf automatische Synchronisation}
\subsection{Implementierung der Mechaniken}
\subsubsection{Client-Side Prediction beim Schießen}
\subsubsection{Interpolation beweglicher Objekte}
\subsubsection{Lag Compensation beim Hit-Scan}
\subsection{Technische Herausforderungen}
\subsubsection{Simulation von Latenz und Paketverlust}
\subsubsection{Fehlerquellen und Debugging}
\subsection{Visualisierungstools}
\subsubsection{Overlays: Prediction, Lag, Interpolation}

\clearpage

\section{Evaluation und Experimente}
\subsection{Testaufbau}
\subsubsection{Szenarien mit simulierten Netzwerkbedingungen}
\subsection{Beobachtungen}
\subsubsection{Einfluss auf Spielgefühl und Kontrolle}
\subsection{Vergleich verschiedener Konfigurationen}
\subsubsection{An/aus-Schalten von Prediction und Interpolation}
\subsubsection{Unterschiedliche Latenzstufen}
\subsection{Diskussion}
\subsubsection{Stärken und Schwächen der Mechaniken}
\subsubsection{Relevanz je nach Anwendungsszenario}

\clearpage

\section{Übertragbarkeit auf andere Anwendungsbereiche}
\subsection{Weitere Echtzeitanwendungen}
\subsubsection{Beispiel: Kollaborative Musiksoftware}
\subsection{Anpassung der Mechaniken}
\subsubsection{Prediction vs. Genauigkeit}
\subsubsection{Verschiebung der Prioritäten: Qualität vs. Reaktionszeit}

\clearpage

\section{Fazit und Ausblick}
\subsection{Zusammenfassung}
\subsection{Ausblick}
\subsubsection{Mögliche Erweiterungen (z.B. Rollback, KI)}

\clearpage

\appendix
\section*{Anhang}
\subsection*{A.1 Code-Snippets}
\subsection*{A.2 Screenshots und Visualisierungen}
\subsection*{A.3 Testdaten}

\section*{Literaturverzeichnis}

\end{document}
