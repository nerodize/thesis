\documentclass{article}

\usepackage{listings}
\usepackage{xcolor}

\definecolor{codegreen}{rgb}{0,0.6,0}
\definecolor{codegray}{rgb}{0.5,0.5,0.5}
\definecolor{codeorange}{rgb}{1,0.49,0}
\definecolor{backcolour}{rgb}{0.99,0.99,0.99}

\lstdefinelanguage{CSharp}{
    language=C,
    morekeywords={
        abstract, as, base, bool, break, byte, case, catch, char, checked, class, const,
        continue, decimal, default, delegate, do, double, else, enum, event, explicit, extern,
        false, finally, fixed, float, for, foreach, goto, if, implicit, in, int, interface,
        internal, is, lock, long, namespace, new, null, object, operator, out, override, params,
        private, protected, public, readonly, ref, return, sbyte, sealed, short, sizeof, stackalloc,
        static, string, struct, switch, this, throw, true, try, typeof, uint, ulong, unchecked,
        unsafe, ushort, using, virtual, void, volatile, while, var, dynamic, get, set
    },
    morecomment=[l]{//},
    morecomment=[s]{/*}{*/},
    morestring=[b]",
    sensitive=true
}

\lstdefinestyle{csharpStyle}{
    language=CSharp,
    backgroundcolor=\color{backcolour},   
    commentstyle=\color{codegray},
    keywordstyle=\color{codeorange}\bfseries,
    numberstyle=\tiny\color{codegray},
    stringstyle=\color{codegreen},
    basicstyle=\ttfamily\footnotesize,
    breaklines=false,                 
    captionpos=b,                    
    keepspaces=true,                 
    numbers=left,                    
    numbersep=10pt,                  
    showspaces=false,                
    showstringspaces=false,
    showtabs=false,                  
    tabsize=2,
    xleftmargin=0.5em,
    frame=none
}

\usepackage[utf8]{inputenc}
\usepackage[T1]{fontenc}
\usepackage[colorlinks=true, linkcolor=black, urlcolor=blue]{hyperref}
\usepackage{listings}
\usepackage{xcolor}

\title{Netcode in einer Unity basierten Zielumgebung}
\author{Alexander Seitz}
\date{14.05.2025}

\begin{document}

\maketitle
\tableofcontents
\newpage

\section{Einleitung}
\subsection{Motivation}
\subsection{Zielsetzung der Arbeit}
\subsection{Aufbau der Arbeit}

\section{Theoretische Grundlagen}
\subsection{Netzwerke in Echtzeitanwendungen}
\subsubsection{Latenz, Paketverlust, Jitter, Tickrate}
\subsection{Architekturen in Multiplayer-Systemen}
\subsubsection{Client-Server-Modell}
\subsubsection{Peer-to-Peer-Kommunikation}
\subsubsection{Hybride Topologien}
\subsubsection{Authoritätsmodelle}
\subsection{Netzwerktheorie}
\subsubsection{Tickrate und Input Delay}
\subsubsection{Reconciliation, Prediction, Interpolation}
\subsection{Mathematische Grundlagen für Bewegung und Synchronisation}
\subsubsection{Vektoren, Transformationen und Rotationen}
\subsubsection{Quaternionen und ihre Bedeutung in 3D-Rotation}
\subsubsection{Interpolationsverfahren}
\paragraph{Lineare Interpolation}
\paragraph{Spherische Interpolation}

\end{document}
